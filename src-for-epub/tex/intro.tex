
Venerable Ajahn Chah was a master at using the apt and unusual simile to explain points of Dhamma. Sometimes he would make an abstract point clear with a vivid and simple image; sometimes he'd tease out the implications of an image in a way that suggested many layers of meaning, offering food for continued thought. In other words, some of his similes provided answers, whereas others provoked questions. 

Since his death, several collections of similes have been drawn from his Dhamma talks. The present translation is based primarily on a collection compiled by one of his Thai students, Ajahn Jandee, in the early years of this decade. I say `primarily' because I have introduced the following changes: 

Three of the similes in the original collection have been replaced by three others, drawn from the talk, `Disenchanted with What You Like' (\textit{Byya khawng thii chawb}); `Bottled Water, Spring Water'; `The Fence'; and `In the Shape of a Circle'. In two of these cases, the original similes were redundant with other similes in the collection. In one, the original simile was more of historical than of practical interest. 

One of the original similes -- `Water Drops, Water Streams' -- includes a few extra sentences from the Dhamma talk in which it appeared.

Some of the titles for the similes have been changed to work more effectively in English.

The order of the similes has been changed to provide a more organic sense of unity and flow.

Ajahn Jandee transcribed his collection directly from recordings of Ajahn Chah's talks with minimal editing, and I have tried to follow his example by giving as full and accurate translation as I can. The unpolished nature of some of the similes is precisely what reveals unexpected layers of meaning, making them so provocative, and I hope that this translation succeeds in conveying some of the same unfinished, thought-provoking quality in English as well. 

Several people have looked over the original manuscript and have provided helpful recommendations for improving it. In particular, I would like to thank Ajahn Pasanno, Ginger Vathanasombat, and Michael Zoll.

May all those who read this translation realize Ajahn Chah's original intention in explaining the Dhamma in such simple and graphic terms.
\bigskip

{\par\raggedleft\noindent Thanissaro Bhikkhu \\
METTA MONASTERY \\
October, 2007 \par}
