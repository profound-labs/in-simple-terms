
{\pagestyle{empty}\enlargethispage{-2\baselineskip}

{\raggedleft\setlength{\parskip}{1em}\setlength{\parindent}{0em}

\vspace*{5\baselineskip}
{\fontfamily{shaker2-light}\fontseries{m}\fontshape{n}\selectfont\titlepagetitlesize\color{textbody} \MakeUppercase{\bookTitle}%
\par\vspace*{\baselineskip}\vspace*{-0.1\baselineskip} {\titlepageauthorsize 108 Similes for Contemplation \\
by Venerable Ajahn Chah

Translated from the Thai \\
by Thanissaro Bhikkhu

}}

\vfill

{\fontfamily{shaker2-light}\fontseries{m}\fontshape{n}\selectfont%
Published for free distribution by\\
ARUNA PUBLICATIONS

This book is available for free download at

\vspace*{-0.8\baselineskip}%
{\titlepagelinksize\href{http://www.forestsanghapublications.org/}{www.forestsanghapublications.org}}%

}

}

\mbox{}\newpage\thispagestyle{empty}%
{\smaller\setlength{\parskip}{0.8em}\setlength{\parindent}{0em}%
{\raggedright%

% \bookTitle \\
% by \bookAuthor

Published by:

Aruna Publications,\\
Aruna Ratanagiri Buddhist Monastery,\\
2 Harnham Hall Cottages,\\
Harnham, Belsay,\\
Northumberland, NE20 0HF\\
UK

Contact Aruna Publications at \href{http://aruno.org}{www.aruno.org}\\
This book is available for free download at\\
\href{http://forestsanghapublications.org/}{www.forestsanghapublications.org}

ISBN \bookISBN

Copyright \copyright\ 2011 HARNHAM BUDDHIST MONASTERY TRUST

\vfill

{\tiny

This work is licenced under the Creative Commons Attribution-NonCommercial-NoDerivs 2.0 UK: England \& Wales Licence. To view a copy of this licence, visit:\\
\href{http://creativecommons.org/licenses/by-nc-nd/2.0/uk/}{http://creativecommons.org/licenses/by-nc-nd/2.0/uk/}\\
Or send a letter to: Creative Commons, 444 Castro Street, Suite 900, Mountain View, California, 94041, USA.

See page \pageref{cc-details} for more details on your rights and restrictions under this licence.

Material included in this book has been previously published by Abhayagiri Monastic Foundation, reprinted here with permission.

\copyright\ 2007 ABHAYAGIRI MONASTIC FOUNDATION

% ISBN 1234567890123

Cover photo by Chinch Gryniewicz.

% Leaf drawing by Ajahn Vimalo.

Produced with the {\fontfamily{cms}\selectfont\LaTeX} typesetting system. The body-text is typeset in Gentium, distributed with the SIL Open Font Licence by SIL International.

\editioninfo

Printed in Malaysia by Bolden Trade.

}

}}



\cleardoublepage

\mbox{}\vspace*{-\headsep}\vspace*{-1.5\baselineskip}

\noindent%
\begin{minipage}[c][\textheight][c]{\paperwidth}

\noindent\hspace*{-20mm}%
\begin{minipage}{\paperwidth}
\centering

{\itshape `\ldots{} The Dhamma is just like this, talking in similes,\\
because the Dhamma doesn't have anything.\\
It isn't round, doesn't have any corners.\\
There's no way to get acquainted with it except through\\
comparisons like this. If you understand this,\\
you understand the Dhamma.'
\bigskip

`Don't think that the Dhamma lies far away from you.\\ It lies right with you; it's about you. Take a look.\\ One minute happy, the next minute sad, satisfied,\\ then angry at this person, hating that person:\\ it's all Dhamma \ldots{}'
\vspace*{2\baselineskip}}

{\smaller\MakeUppercase{Ajahn Chah}}

\end{minipage}

\end{minipage}

\clearpage\thispagestyle{empty}


\mbox{}\vspace*{-\headsep}\vspace*{-2.5\baselineskip}\vspace*{2em}

\noindent%
\begin{minipage}[c][\textheight][c]{\paperwidth}

\noindent\hspace*{-20mm}%
\begin{minipage}{\paperwidth}
\centering\small

We would like to acknowledge the support \\
of many people in the preparation of this book, \\
especially to the Kata\~n\~nut\=a Group of Malaysia, Singapore \\
and Australia, for bringing it into production. 
\bigskip

\end{minipage}

\end{minipage}


}
